\RequirePackage{fix-cm}
\documentclass{Latex/svjour3}
\usepackage{mathptmx}      % use Times fonts if available on your TeX system
\usepackage{latexsym}
\journalname{Sloan Analytics Conference}
\usepackage{setspace}
\usepackage{url}
\usepackage{multirow}
\usepackage{graphicx}

%\linespread{2}

\begin{document}

\title{Hot heads, cool heads, and tacticians: Measuring the mental
  game in tennis}

\titlerunning{Measuring the mental
  game in tennis}  

\authorrunning{S. Kovalchik, M. Ingram}

\author{Stephanie Kovalchik\and Martin Ingram}

\institute{S. Kovalchik \at 
              Tennis Australia, Melbourne Park, Australia \\
              \email{s.a.kovalchik@gmail.com}      \\
           \and
           M. Ingram \at
             Stratagem Technologies, London, United Kingdom \\
             \email{martin.ingram@gmail.com}      
}

\date{}


\maketitle

\begin{abstract}
It is often said that winning in tennis is determined as much by a
player's mental game as his or her physical ability yet quantifiable
metrics for the mental side of the game are lacking. We present an
approach to identify mentalities in tennis with dynamic response
patterns that quantify how players probability of winning a point vary
in response to the changing situations of a match. Using nearly 3
million points played by professional male and female tennis players between
2011 and 2015, we found that top players were
generally affected by the state of the score and a variety of pressure
situations, exhibiting hot hand effects when up in the match,
defeatist effects when down, and performing less effectively in clutch
situations than on less important points. However, player-to-player
variation in these dynamic effects revealed that not all players
shared the average mentality profile. The distinct ways in which
players deviated from the field highlighted a variety of differences
in how players respond to pressure and point history and suggested a
diversity of player mentalities at the elite
level. Accounting for these dynamic changes in performance were shown
to improve the prediction of match outcomes for the men's and women's
games, substantiating the importance of mentality for performance in tennis.



\keywords{Tennis\and Player evaluation\and Mixed modeling\and Hierarchical clustering}
\end{abstract}

\section{Introduction}

Mentality is an essential ingredient of all athletic
performance. However, the importance of a player's mental skills in
deciding the outcomes of competition or how fans experience
competition varies widely across sports. Tennis, the
most popular individual sport in the
world\footnote{\url{http://www.biggestglobalsports.com}}, is
frequently said to be as much about the mind as the body. Indeed, Jimmy Connors, holder
of 109 career titles, has gone so far as to estimate that 95\% of tennis is a mind
game\cite{samulski2007tennis}. Despite these claims, the mental side
of tennis has received limited scientific study and there is currently
a lack of quantitative
evidence about the mentalities of today's top tennis players.

Prior work attempting to measure the mental aspect of elite
athletic performance has been limited for all sports let alone
tennis, reflecting the challenges of
measuring the mental game. Prior approaches have largely been
qualitative in nature, relying on interviews\cite{young2011understanding} or surveys\cite{taylor1987predicting} of players and
coaches to gain insight about the mental game. These studies have the
drawback that they presuppose that salient mental skills can be
measured with a questionnaire or elicited from a conversation with
coaches and players. Such studies have also not taken advantage of the
many years of historical performance data and what it can reveal about the mental game. 

In this paper, we present a novel quantitative method to investigate the mentality
of professional tennis players from observed match performance. Our approach is based on the premise
that tennis player's
reveal aspects of their mental skills in the way they respond to
changes in the state of play over the course of match, i.e. the
dynamics of a match. Using nearly 3 million points played in
professional singles matches for the men's and women's tours between 2011 and 2015, we quantify player
response patterns to point dynamics, identify common mentalities, and
evaluate the importance of mentality for match performance. 


\section{Methods}

\subsection{Data}

Point-level data was obtained for singles matches played on the ATP
and WTA tours during the 2011 to 2015 seasons. For the ATP tour, matches
were restricted to those played at tournaments in the 250 series or above, which
includes the 9 Masters 1000 events and 4 Grand Slams (Australian Open,
French Open, Wimbledon, and US Open),
where players have the opportunity to earn the most ranking points and
prize money during the season. The WTA data included matches from all
International, Premier, and Grand Slam tournaments. To ensure an
adequate sample size of points for each player, only players with 3 or
more match appearances were eligible for inclusion. Point-level data
was obtained from the Tennis Abstract\footnote{Github source: \url{github.com/JeffSackmann/tennis_pointbypoint}} and accessed with functions from
the R package \texttt{deuce}\footnote{Github source: \url{github.com/skoval/deuce}} .

The final datasets for the present paper included over 1.6 million
points across 10,101 matches played by 434 players for the ATP tour
and 1.4 million points for 424 players and 9,668 matches
for the WTA tour (Table~\ref{tab:description}). Approximately 20\% of
the points in these datasets were played at the Grand Slams. 


\begin{table}[ht] \centering
\caption{Summary of Point-level Datasets of ATP and WTA
  Singles Matches, 2011-2015}
\begin{tabular}{l cc}
  \hline
Variable  [Shorthand] & ATP & WTA \\ 
  \hline
Points & 1,610,439 & 1,373,095 \\ 
  Players& 434 & 424 \\ 
  Matches & 10,101 & 9,668 \\ 
  Grand Slam Matches & 1,869 & 2,066 \\
\textbf{Candidate Predictors} & &  \\
  ~~Tiebreak, \% & 3.4 & 1.9 \\ 
  ~~Break point, \%& 8.4 & 11.7 \\ 
 ~~Point away from break point, \% [Break point -1] & 17.6 & 21.3 \\ 
  ~~Set or more up, \% [Set+] & 22.4 & 20.8 \\ 
  ~~Set or more down, \% [Set-] & 23.8 & 21.5 \\ 
  ~~Player won last point, \% [Just won] & 53.8 & 51.4\\ 
 ~~Serve game after missed break, \%  [Missed break, serve] & 10.1 & 10.0 \\ 
  ~~Return game after missed break,\%  [Missed break, return] & 9.1 & 8.9 \\ 
  ~~Importance, Mean (SD) & 0.05 (0.05) & 0.06 (0.05) \\ 
  ~~Last game's points, Mean (SD) & 5.5 (2.9) & 5.8 (3.2)\\ 
  ~~Point spread, Mean (SD) & -0.38 (4.5) & -0.19 (4.9) \\  \hline 
\end{tabular}
\label{tab:description}
\end{table}

The analytic datasets were divided into training and validation data
for the purpose of model development and testing. The validation data
were the points played in matches at the 2015 Grand Slam
tournaments, the most recent majors at the time of writing. The training data for each Grand Slam were all points
played up to but not including the Grand Slam event. Thus, four
different training and testing datasets were constructed for each
tour and all tournaments included in the training were independent of
the validation tournaments. The validation data included 105,717
points for the ATP and 71,341 for the WTA.


\subsection{Player Dynamic Model}

We introduce a player dynamic model (PDM) to quantify how an individual
player's performance is uniquely affected by the conditions of a point
in a match. The player-specific dynamics estimated from the PDM will
be the basis for characterizing player mentality. The dependent
variable of the model was the win-loss outcome for a point, with
respect to the server of that point. Let $y_{ijk}$ be the point outcome
(Win = 1, Loss = 0) for the $i$th player serving against the $j$th opponent, on the $k$th service
point. 

The PDM describes the relationship between the conditions of
the point and the point outcome with the following linear mixed
probability model, 

\begin{equation}
E\lbrack y_{ijk} \rbrack = (\alpha_{i} + \beta_{j} + \theta)' \mathbf{X}_{ijk}.
\label{eq:pdm}
\end{equation}

\noindent The vector $\mathbf{X}_{ijk}$ contains an intercept, which
defines the baseline serve and return ability of the player and opponent, and $p$
dynamic features that affect performance on the point outcome (e.g. being a
set down, facing a break point, etc.). The parameters $\theta$ are
fixed dynamic effects that represent how player's are affected by
conditions on average. The parameters $\alpha_i$ and $\beta_j$ are
player random effects for each feature for the $i$th player who is
serving and $j$th
player who is returning and each are drawn from  from a
multivariate normal distribution with zero mean and general variance
structure. The player random effects allow
that some players could be more or less affected by the state of a
point and that these effects could additionally depend on whether a player
is serving or returning. The combination of the server and returner
effects determine the expected win probability for the
point. 

\textit{Remarks.} The PDM can be viewed as an extension of two
established models for predicting point outcomes in tennis. When the
feature matrix $\mathbf{X}$ includes only an intercept, the PDM
becomes a regression model for the opponent-adjustment proposed by Barnett and
Clarke. Klaassen and Magnus also proposed a dynamic model to test for
deviations from the IID model, which says that points in a match are
independent and identically distributed. In contrast to the PDM, the model of Klaassen and Magnus considers a limited number of
dynamic effects and does not incorporate player-specific
effects, which are the parameters of primary interest in the present work.

\subsection{Predictors}

We fit the PDM with eleven candidate predictors that are listed in
Table~\ref{tab:description}. The candidates included a range of dynamic
point, game, and set conditions. The majority of the point conditions
focus on various types of pressure situations, including
indicators of whether a point occurs during a tiebreak, is a
break point opportunity for the returner, or a point away from a break
point opportunity (Break point -1). These predictors can be considered types of
important points because they have a greater influence on the game or
set outcomes than other points. We also include a probabilistic measure of
point importance, defined by Morris\cite{morris1977most}, that is equal to the
average change in match win probability when the current point is won
compared to win it is lost, given the state of the match at this
point. One final point
condition is an indicator of whether the server won the previous point
(Last won), which captures short-term correlation between points that
could arise from a one-point hot hand effect, for example. 

Three predictors contained information about game history. Two
concerned missed opportunities to break service and how this affected
the subsequent service game (Missed break, serve) and return game
(Missed break, return). In the coding for these predictors, it is
assumed that the psychological impact of a missed break is a
within-set effect and does not carry over into the games of other sets
or tiebreaks.  We
also examined how the number of points played in the previous game
might influence play in the current game. A more closely contested
game will have more points played, and could serve as an indirect
measure of player fatigue. 

There were also three set conditions included in the set of
predictors. Two of the predictors indicated when the player serving
was either up a set or more (Set+) or down a set or more (Set-) in the
match. Finally, in order to capture longer term
momentum effects than those due to the outcome of the previous point,
we tracked the point spread across games within a set, subtracting the
points won in the set by the player serving from those of the player
returning. 


\subsection{Model Estimation}

The PDM was implemented with the R  package \texttt{lme4} using a
Gaussian family for the outcome distribution. The Gaussian
model has a number of desirable features. Most importantly, the
dynamic effects have a simple and meaningful interpretation, as they
represent the absolute change in point
win probability for a
one unit increase in a feature. The model is also the most
computationally efficient within the generalized linear
family. However, the model is most appropriate for continuous
outcomes, whose mean, unlike a binary outcome, does not have a
constrained support. 

Klaassen and Magnus
have previously shown that the linear model works well in practice for modeling
point win probabilities. We
also conducted our own investigation by obtaining marginal
predictions for the dynamic effects using a logistic model and
comparing these to their corresponding effects with the linear
probability model. The effects differed by no more than one
significant digit, indicating that the lack of constrained estimation
had a negligible impact on the PDM estimates.

In addition to the PDM, we fit two additional models for the purpose
of evaluating the overall improvement in prediction performance with
the inclusion of player-specific dynamics. One of these was a simpler
version of the dynamic model that had average effects for the eleven
dynamics and only an intercept term for the player effects. We will
refer to this model as the average dynamic model (ADM). The second
model had only an intercept term for the average and player
effect. Without any dynamic effects, this model is equivalent to an
IID model. 

\subsection{Model Performance}

To summarize the overall magnitude and significance of the dynamic
effects, we fit the ADM with all
of the available data and computed 95\% confidence intervals for each
effect.  The added value of including player-specific effects for each
dynamic feature was measured by the change in the Akaike Information
Criterion (AIC). The AIC is a measure of overall model fit that
allows comparisons between non-nested models and awards more parsimonious models. For each
dynamic predictor, we estimated the change in AIC with the inclusion
of player-specific effects versus a constant effect for all players. 

To assess the implications of mentality on points for predicting the
outcomes of matches, we used a Monte Carlo simulation to test the
predictive performance of the PDM. The simulator ran $5,000$ trial
matches for each pairing of the 2015 Grand Slams (a best-of-five
format for the ATP and best-of-three for the WTA). Each simulated point on serve
in a match adjusted the server's probability of winning the point
according to the state of that point as described by the PDM, and the
fraction of trial matches won gave an estimate of the probability that
a player won the match. The
player dynamic match predictions were compared to two alternative
models: the ADM that assumes constant dynamic effects for all players
and an IID model that assumes no dynamic effects, i.e. a constant
point win
probability for each server. Our primary metric of performance was
log loss, as it places a high penalty on overconfident predictions\cite{yuan2015mixture}. 



\subsection{Identifying Mentality Types}

The ways in which a player's performance is affected by the conditions
of a match provide insight into the player's mentality. The set of
player-specific dynamic effects from the PDM\textemdash which represent how much
more or less a player's performance on a point shifts in response to
conditions compared to the field\textemdash provide a mentality
profile. To identify common mentalities on the tour, we applied a
hierarchical clustering method to the dynamic profiles of the players
who competed in the 2015 Grand Slams. Only the most
salient features for distinguishing player types were included, which
were defined as the features that showed an AIC
improvement over the ADM that indicated significant variation in the effect from one player to the
next. 

Prior to clustering, the dynamics effects of the mentality profiles
were converted from their probability scale to a z-score so that each
effect would have the same mean and variance and allow for direct
comparison between effects. A distance measure was then applied to all
possible pairs of the standardized profiles (e.g. Player A's
standardized dynamics on tiebreaks, break points, etc. versus Player
B), which results in a dissimilarity matrix. We then apply a linkage
approach to identify clusters among players. 

There are a number of options available for both the distance metric
and linkage approaches and the choice of method in each case could potentially influence
the resulting clusters. Three common measures of distance are the
Euclidean ($L_2$-norm), the Manhattan ($L_1$-norm), and absolute
maximum (or Chebyshev's distance). The methods primarily differ in
their response to outliers with the maximum distance being most
sensitive to extremes and the correlation least sensitive.  

The linkage method is the technique applied to the pair-wise distances
to determine a measure of the cluster distance. \textit{Single
  linkage} is a nearest neighbor method that assigns clusters based on
the minimum distance. Average linkage chooses the cluster that
minimizes the average distance between. Complete linkage is the
opposite extreme of single linkage in that it assigns clusters by
maximizing the difference between clusters. Each of the hierarchical
techniques begin with all units assigned to a single cluster, i.e. a
`bottom up' approach. 

There is no universally best method among these; the performance,
instead, depends on the properties of the data used
\cite{kumar2014performance}. For this reason, each
combination of distance measure and linkage measure were examined. The
approach selected was one that showed the largest number of patterns
containing two or more players per cluster. 

The choice of the number of clusters was selected by beginning with a
large number ($K = 12$) and visually inspecting the patterns of each
cluster with parallel coordinate plots. When two or more clusters
could not be easily distinguished the number of clusters was reduced
by one. This process was repeated until all patterns could be
uniquely described. 



\subsection{Player Unpredictability}

It is possible that some players will not easily fit into any of the
identified mentality types. One way this could arise is if a player's
shifts in performance are essentially random, ups and downs that are
unrelated to the specific conditions of the point. To examine which
player's exhibited more or less of this kind of volatility in their
mentality, we computed the mean Brier score of the PDM predictions for each player on serve and
return when applied to the Grand Slam validation data. Outliers were flagged
as players with variances that were 2 or more standard deviations from
the mean. 

\section{Results}

\subsection{Characteristics of Dynamic Predictors}

Among the eight categorical dynamics, a
server's win on the previous point was the most common, occurring
slightly more than half of the time for each tour
(Table~\ref{tab:description}). For one of every five points on serve,
at least one competitor would be expected to be a set up or down in a
match based on recent years of matchplay. A
similar percentage of points would be one point from a break point
opportunity. Break points and points played in games following a
missed break opportunity were some of the least common events among
the predictors, happening approximately 10\% of the time during a
match. The rarest dynamic event was a tiebreak, which occurred
for 3\% of points on the men's tour and 2\% of points on the women's
tour. 

The average importance of a point in a tennis match was 5 probability
points, corresponding to an expected increase in match win probability
of 5 percentage points when the point was won versus lost
(Table~\ref{tab:description}). In recent years, we also found that the average number of points
played in a game was 6 for both tours. The average point spread over
the games in a set was approximately zero but it was not unusual to
observe differences as large as 10 points. 

\subsection{Average Dynamic Effects}

The average dynamic effects on point outcomes identified factors that significantly increased the
advantage of the server as well as effects that increased the
advantage of the returner. For both tours, a server's win opportunity was negatively
affected when playing a set down, when facing a break point, or when
facing more important points overall (Table~\ref{tab:estimates}). By contrast, when a server was ahead a set (or more)
in a match, had won the previous point, or otherwise had a lead in the point
spread, we found the server generally had a significantly greater
probability of winning a point. Because the point spread can take
negative values when the server is behind in the set, the impact of
spread would, in this case, have a comparable negative effect on the
server's win probability. In data not shown, we assessed the linear
assumption for point spread and found that the relationship was best
described by a line with approximately equal slope for positive and
negative spreads with respect to the player serving.

\begin{table}[ht] \centering
\caption{Average Effects of Dynamic Conditions on Point Performance
  Played for the 2011 to 2015 ATP and WTA Tours}
\begin{tabular}{l ccc ccc} \hline
\multirow{2}{*}{Dynamic} & \multicolumn{3}{c}{ATP} & \multicolumn{3}{c}{WTA} \\
 & Estimate & 95\% CI & $\Delta$AIC$^a$ & Estimate & 95\% CI &$\Delta$AIC\\ \hline
Base rate & 63.20 & (62.72, 63.69) &--& 55.79 & (55.31, 56.27)  &\\ \noalign{\smallskip}
Tiebreak& -0.06 & (-0.53, 0.41) &2.8& -0.61 & (-1.27, 0.05) &-2.3\\ 
Break point & -0.76 & (-1.08, -0.44) &0.9& -0.63 & (-0.95, -0.31) &0.2\\
Break point -1 & -0.39 & (-0.61, -0.18) &8.9& -0.02 & (-0.24,
0.21) &0.6\\ 
Set+ & 1.43 & (1.24, 1.63) &5.2& 1.71 & (1.49, 1.92)  &-0.4\\ 
Set- & -1.94 & (-2.13, -1.75) &19.2& -2.01 & (-2.22, -1.80)  &0.4\\
Importance & -4.34 & (-6.43, -2.26) &32.2& -5.56 & (-7.77, -3.36)  &0.6\\ 
Point spread  &  0.30 & (0.28, 0.32) &13.3& 0.31 & (0.29, 0.33) &-2.5\\
Just won &0.63 & (0.47, 0.78) &44.4& 0.51 & (0.33, 0.69) &0.6\\ 
Missed break, serve  &0.21 & (-0.09, 0.51) &-4.1& 0.20 &(-0.12, 0.52)&-2.6\\ 
Missed break, return   & -0.22 & (-0.49, 0.04) &-3.4& 0.12 & (-0.18, 0.41)  &-3.3\\ 
Last game's points & -0.01 & (-0.04, 0.02) &-5.3& -0.02 &
(-0.05, 0.01) &  0.4 \\ \hline
\multicolumn{7}{l}{CI = Confidence interval, AIC = Akaike Information
Criterion} \\
\multicolumn{7}{p{5in}}{a Change in AIC with inclusion of player-specific
  dynamic effects compared to a constant dynamic effect. Larger values
  indicate the player-specific model provided  a better fit to
  observed point outcomes.} \\
\end{tabular}
\label{tab:estimates}
\end{table}

While the majority of the effects were nearly identical for both the
men's and women's tours, there were a few interesting
exceptions. Being a point away from a break point opportunity had a
modest negative effect on the server's advantage among male players
but not female players (Table~\ref{tab:estimates}). For the men's
tour, missed break
opportunities had a modest boost to the next return game of the
returner who failed to convert but no evidence of an effect for the
women's tour. However, we found some evidence of a decrease in serve
advantage for the women's tour during tiebreak points, which was not
observed for the men's game. 

\begin{figure}
\includegraphics[scale=0.8]{figs/avg_effects.pdf}
\caption{Average dynamic effects on the probability of winning on
  serve based on point-level data for 2011-2015 ATP and WTA singles
  matches. The y-axis shows the estimates change in the probability of winning a
  point on serve (in percentage points) for a one standard deviation
  change in the dynamic predictor. Error bars denote the 95\%
  confidence interval for the estimated effect.}
\label{fig:1}
\end{figure}

The estimates in Table~\ref{tab:estimates} represent the estimated change in
serve probability (in percentage points units) associated with a one-unit
increase in a dynamic factor. Because a one-unit increase might not
be meaningful for all of the predictors (e.g. point importance), we
compare the average effects on a standardized scale where each bar
corresponds to the change in serve win probability for one standard
deviation increase in the corresponding dynamic factor
(Figure~\ref{fig:1}). This plot reveals that the strongest predictor
for both tours was point spread, where a server with a one standard
deviation lead in the point score was estimated to have a 1.4-1.5
percentage point increase in point win probability. Being a set up or set down were
runners up with a roughly 1 percentage point effect size, a server
being a set up
adding to the server's advantage and server being a set down adding to
the returner's advantage. Other game and point conditions
had more moderate effects, with some indication of a relatively stronger
negative effects on importnat points and tiebreak points for female
players compared to male players.

\subsection{Player Dynamic Effects}


The estimates shown in Table~\ref{tab:estimates} represent the effect of each dynamic feature if all players were equally
influenced by the state of the match. However, because not all players
have the same mentality on court, we would expect some players to respond differently to point conditions than others. The player dynamic
model allows for player-to-player differences in their response to the
state of the match by
estimating a separate dynamic effect for each player. 

To determine
when the player-specific dynamics better explained observed
performance, we calculated the improvement in AIC with the PDM. The changes in
AIC shown in Table~\ref{tab:estimates} reveal the presence of
important player-to-player differences in the dynamic effect, positive
changes reflecting a better fit with the PDM. For the men's game, all but three of the predictors demonstrated
important player dynamics. The exceptions were both indicators of missed
breaks of service and the total points played in the last game. While
dynamic factors with a significant average effect were generally found
to also have important player-to-player variation, tiebreak points
were found to have important player-to-player variation in performance
despite a weak average effect for the men's tour. 

For the WTA, improvements with the PDM were fewer and smaller in
magnitude than for the ATP. Six of the 11 factors\textemdash break
points, being one point from a break point, being down a set, point
importance, the outcome of the previous point, and the total points
played in the previous game \textemdash showed important
player-to-player variation (Table~\ref{tab:estimates}). Thus, male
player responses to tiebreak points, being a set up, and having a lead
in point were more variable than for female players, whereas female
player responses to the total points played in a game were more variable. 

\subsection{Model Performance}

On the ATP, the importance of mental effects
was confirmed by the greater accuracy and lower log loss for the
predictions of the dynamic models compared to the IID model (Table
\ref{tab:performance}). The consistent superior performance of the PDM
over the ADM substantiates the importance of player differences in
response to the changing situations of match play on the men's tour.

While the dynamic models also improved match predictions for the
women's tour, the differences were smaller than for the men's and the
performance of the average and player-specific dynamic models were
statistically equivalent. 

\begin{table} \centering
\caption{Summary of Match Prediction Performance for the 2015 Grand Slams}
\begin{tabular}{l ccc ccc} \hline
\multirow{2}{*}{Tournament} & \multicolumn{3}{c}{Accuracy} & \multicolumn{3}{c}{Log Loss} \\ 
&	IID &ADM &PDM &IID &ADM &PDM \\ \hline
\textbf{ATP} & & & & & & \\
~~Australian Open	&73.3&	74.1&	75.9&	0.516&	0.505	&0.491\\
~~French Open	&70.6&	71.4&	73.9&	0.552&	0.541	&0.532\\
~~US Open	&75.5&	75.5&	76.4&	0.507&	0.496&	0.493\\
~~Wimbledon	&73.3&	72.5&	72.5&	0.540&	0.523&	0.517\\
~Overall	&73.1&	73.3&	74.6&	0.529&	0.517&	0.509                    \\
\textbf{WTA}  & & & & & & \\
~~Australian Open	&74.0&	73.2&	73.2&	0.595&	0.573&	0.576\\
~~French Open	&72.7&	71.1&	71.9&	0.568&	0.561&	0.563\\
~~US Open	&65.8&	65.0&	64.2&	0.663&	0.632&	0.634\\
~~Wimbledon	&70.4&	70.4&	71.2&	0.568&	0.555&	0.560\\
~Overall	&70.8&	69.9&	70.1&	0.598	&0.580& 0.583\\ \hline
\multicolumn{7}{p{3.8in}}{IID = Independent identically distributed, ADM =
  Average dynamic model, PDM = Player dynamic model}
\end{tabular}
\label{tab:performance}
\end{table}

\subsection{Mentality Profiles}

\subsubsection{Men's Tour}

The eight dynamic factors that improved the predictive performance of
point outcomes for the men's tour revealed eight unique mentalities
among the male players who had competed in one or more of the 2015
Grand Slams. The players with each mentality type are displayed in
Figure~\ref{fig:atp_dendro} as a dendrogram in which mentalities that
are more similar are closer together in their order from top to
bottom. The underlying feature profiles for each group are displayed in
Figure~\ref{fig:atp_coord}. Here, each line is a specific player's
set of dynamic effects on serve and return and effects are scaled so
that they all have an equal standard deviation of one. A smoothed
regression line is plotted over the observed profiles in each panel to
highlight the key differences from the status quo (`The Field') shown
in gray.

\textit{The Field}. We begin with a description of the mentality
suggested by the cluster with the largest number of players and,
consequently, the most common profile among top male players. This
group exhibits a drop in performance when pressure is on the serve, as
indicated by the negative effects when a set down and facing important
points, including break points and tiebreaks (Figure~\ref{fig:atp_coord}). These players also
exhibit sensitivity to the state of the point score, as is indicated
by the positive effects on winning the previous point, having an edge
in point spread, or being a set up. These `hot hand' effects induce a
corresponding loser's curse on the return game, in which players who
fall behing are even less likely to win a point than when even or
ahead in the score. 

\textit{John Isner}. One of two players with a unique profile was big
server John Isner. The large positive effects on serve indicate
greater overall mental toughness when serving than any other player
evaluated. On the defense game, Isner's mentality showed a lack of
confidence on break point and other important points. His performance
on the return game was otherwise similar to the field, with the
exception of tiebreaks where he showed strong performance whether
returning or serving. 

\textit{Tiebreak Specialists}. Like Isner, these players shine on
tiebreak points, raising their performance when serving or
returning. On other point types, they also exhibit a similar disparity
between the service and return games, with greater overall confidence
on serve, but to a less extreme degree than Isner.

\textit{Fabio Fognini}. The second player found to have a unique
mentality was Fabio Fognini, Italian No. 1 at the time of this
writing. Fognini's distinctive characteristics backup the mercurial
label he has often been given by the media\footnote{Medlock,
  W. (June 14, 2015) `Ranking the Most Unpredictable Tennis Players
  Today'. Retrieved from: \url{http://bleacherreport.com/articles/2495031-ranking-the-most-unpredictable-tennis-players-today/}}. While being unusually
mentally strong on more important points (especially on the return
game) and on making break point opportunities, the large negative
effects when a point or set down on return indicate that he is one of the
players most susceptible to collapse.

\textit{Champions}. The players who currently hold the most Grand Slam
titles Novak Djokovic, Roger Federer, Andy Murray and Rafael Nadal
(colloquially referred to as `The Big Four') were all found in the
same mentality cluster, suggesting a `Champion's mentality'. The
players in this group exhibit similar strength on serve as the big servers among the
tiebreak specialists except for the boost in performance on
tiebreaks. On the return game, these players set themselves apart with
the mental toughness they show in clutch situations: important points and creating break
point opportunities. While the majority of these players also showed a
greater ability to convert break points than other competitors, it is
interesting to note that Roger Federer was most similar to the field
on this characteristic. 

\textit{Opportunity Makers}. This group of players had many of the
tendencies of the champions group but to a lesser degree. The most
consistent positive trait observed compared to the field was the
tendency to raise their game to make an opportunity to break serve,
shown by the positive effect on the point away from break point on
the return game. Several of the most exciting players today's
game\textemdash Jo Wilfried Tsonga and Gael Monfils\textemdash are
included in this group.

\textit{Tight}. This mentality was the only one that was noteworthy
for being weaker on certain points than the average top
player. Specifically, in clutch situations on serve and when down a
point or a set on the return game, these players showed a greater drop
in win probability than any others. Finding former
World No. 1 Lleyton Hewitt in this group was unexpected but could be
explained by the point-level data only covering the final years of his career.

\textit{Score Keepers}. In addition to being generally less confident
on serve, the final group of players were unique in
their response to the outcome of the previous point, showing a hot
hand response when winning a point on serve and a corresponding `cool
hand' after losing a point on return. Thus, the performance of these
players are unusually sensitive to the short-term state of the score. 

\subsubsection{Women's Tour}

Considering the six player-specific dynamics for the women's tour, eight unique mentalities were also found among players competing
in the 2015 Grand Slams (Figure~\ref{fig:wta_dendro}). 

\textit{The Field}. Like male players, the majority of the service
game of top female
players is negatively affected under pressure but benefits from a
recent point win\textemdash a mini hot hand effect. 

\textit{Fighters}. Several of the top players were found to raise the level of
their play after tightly contested games (Figure~\ref{fig:wta_coord}). These players had large
positive effects associated with more points played in the last game
on both return and service games, but especially on the service
game. These players also exhibit greater cool-headedness in clutch
situations on the return game, but their most unique characteristic is
the fighter's mentality suggested by their improved performance after
long points. It is worth noting that World No. 1 Serena Williams,
known for her mastery of the comeback, had
the largest positive effect for long points on serve. 

\textit{Stoics}. Another group of players
showed even greater cool-headedness on the return game than the
`fighters'. The defense performance for these players was virtual
unaffected by the state of the score or the importance of points other
than break points. On the service game, these players were also the least
negatively affected by pressure and the least phased by being a set
down. Two players often praised for their mental toughness\textemdash
Maria Sharapova and Victoria Azarenka\textemdash were found in this
group.

\textit{Faders}. In sharp contrast to the `fighters' described above,
another set of players had a notable negative effect in their service
game after a closely contested game, which could be the effect of mental or
physical fatigue. 

\textit{Tight}. While nearly all players show some decline in serve
performance in pressure situations, only one group of players had
strong and nearly equal negative effects when facing a break point, a
break point opportunity, or other important points.  Although less
pronounced on the return game, the greater negative effect on points
away from break point suggest that vulnerability in clutch
situations affects both parts of these players' game. 

\textit{Clutch Servers}. We also observed a group of players that were
generally unaffected by pressure on serve, having little or no
effect on break points and other important points. There was also some
evidence of improved performance on serve after long points like that
observed for the `fighters' group. Several players found in this
group, like Sam Stosur and Petra
Kvitova, are known for
inconsistent displays of excellent play.

\textit{Savers}. Two players, Barbora Strycova and Caroline Garcia, stood out from the rest of their cohort
for being unusually unmoved on serve when facing a break point.

\textit{Preemptors}. One of the larger group of players were
noteworthy for their mentality on serve when a point from facing a
break point. Unlike the field, these players tended to increase their
win probability to avoid a possible break of service. Several rising
stars of the WTA tour, including Garbine Muguruza and Belinda Bencic,
were members of this group.


\subsection{Unpredictable Performance}

When we measured the prediction error of the PDM for each player (a
metric of a player's unpredictability), we found more outliers who
were unusually predictable than outliers who were unusually
unpredictable. For both tours, a small
but roughly equal number of players were highly predictable on serve
and return. 

Figure~\ref{fig:volatility} highlights the ten players on
each tour who were the most extremely predictable. On the men's side,
the group clustered in the lower left quadrant are players who had
very little variation on the serve or return game after accounting for
the dynamic effects of the PDM. Notably, 3 of the strongest servers on
tour (John Isner, Milos Raonic, and Ivo Karlovic) were among this
group. The lower right quadrant consisted of players who were
predictable on serve but much less so on return. It was surprising to
observe 3 of the greatest players of the current era (Roger Federer,
Novak Djokovic, and Rafael Nadal) in this group. 

While a similar pattern in predictability was found for the women's
tour, fewer of the outlying players were as highly ranked as for the
men. The exception was for the lower right quadrant where we, as with
the men, we found several of the tour's greatest champions: Serena
Williams and Maria Sharapova. The similarity of this result for the
men's and women's tours makes the intriguing suggestion that mental
steadiness on serve combined with variety on return could be defining
characteristics of a champion at the professional level. 


\section{Discussion}

In this paper, we presented a novel approach to quantify changes in
tennis performance in response to dynamic situations on serve and
return in a match. When applied to millions of points of recent performance data, we found that top players were
generally affected by the state of the score and a variety of pressure
situations, exhibiting hot hand effects when up in the match,
defeatist effects when down, and performing less effectively in clutch
situations than on less important points. However, player-to-player
variation in these dynamic effects revealed that not all players
shared the average mentality profile. The distinct ways in which
players deviated from the field highlighted a variety of differences
in how players respond to pressure and point history and suggested a
diversity of player mentalities at the elite
level. Accounting for these dynamic changes in performance were shown
to improve the prediction of match outcomes for the men's and women's
games, substantiating the importance of mindset for performance in tennis.


A fundamental feature of our approach for measuring mentality is the isolation of a player's response to changing
match situations from his or her overall win ability. This is done by
focusing on how a player's performance varies around his or her own
baseline probability of winning a point. As a
consequence of this separation of baseline ability from point-to-point
variation in performance, the approach allows that players with
different physical skills could still share a common
mentality. 

In spite of this flexibility in the approach, our findings revealed
instances where the physical and
mental sides of the game were closely linked. This was most striking
on the men's tour in the clustering of players known for their big
service game (John Isner, Milos Raonic, etc.) and the similarity in
the profiles of the most dominant players of the tour. The finding
that all of the `Big 4' (Novak Djokovic,
Roger Federer, Andy Murray, and Rafael Nadal) deviated in very similar
ways in response to changes in pressure and point history suggests
the existence of a champion's mentality that is notable for clutch
performance on the return and relative imperviousness to the
conditions of the point on serve. 

Another interesting relationship between player skill level and
point-to-point performance among ATP players was the finding that 3 of
the most accomplished players on the men's game (Roger Federer, Novak
Djokovic, and Rafael Nadal) were also some of the most predictable
players on serve and most unpredictable on return. Considering this
result with the qualities of the champion profile that was
characterized by stoic serve ability and an attacking defensive game,
we conclude that the most successful players on the men's tour distinguish themselves for their
mental toughness [cite] on serve and their variety on the return
game, which challenges the conclusions of prior work stating that
players should `play every point as it comes'\cite{klaassen2001points}.

The average dynamic effects for the men's and women's tour were
remarkably similar, with the only statistical difference being the
women's absence of an average effect on points a point away from a
break point that was found to negatively effect men's serve
performance overall. However, player variation in effects was less extensive on the women's
tour compared to the men's tour. Fewer of the dynamic variables showed
significant player-to-player variation for female players, and the
variation that was observed was largely restricted to the service
game. This suggests that the variety of mentalities on the women's
tour might be less numerous than the men's or that other dynamic
factors not considered in this paper are needed to explain variation
in female point performance. 

When we examined the unexplained variation among female players, we
found that two of the current greatest female players, Serena Williams
and Maria Sharapova, had similar characteristics as the male
champions. These players were some of the steadiest players on serve
but the most variable on return. The observation of this pattern on
both tours adds strength to the conclusion that champions share common
mental skills on court and further suggests that the champion's
profile might be independent of gender.

In interpreting the observed point-to-point changes in performance, we
have regarded these deviations as indicators of shifts in psychological traits of
a player like confidence, mental toughness, or resilience. It is
important to acknowledge other possible explanations for systematic fluctuations in
performance. Shifts of performance with specific match situations could be a
conscious tactic, like a defender who might plan to take bigger risks
when a point from a break point opportunity. Also, because point-level
data is the result of the behavior of two players, it is possible that
observed changes associated with a particular player are owing to the
ways other players adapt to that player's game, like players becoming
tight on big points when facing a top 10 player. 

Even with uncertainty as to the definitive causes of point-to-point
differences in player performance, the measurement of these deviations
could be a valuable aid to coaches by visually highlighting the
relative strengths and weaknesses in how their player's performance
changes with specific match conditions. When a player's profile is
placed in the larger context of the field, it can also suggest players
with similar and different responses to match situations, which could
suggest strategies for modifying a player's game where room for
improvement is needed. Such visual displays as presented her could
also be an engaging way to contrast player performance to fans.

Present efforts are underway to amass shot-by-shot data that would
enable the study of performance to go beyond the simple outcome of the
point when characterizing player performance. Ball and player
position tracking from camera and wearable technologies will also
become increasingly common over time. As richer and more detailed data
on tennis performance becomes available in the future, the framework
we have presented in this paper provides analysts with a valuable tool
for evaluating player mentality at an increasingly precise level of detail.


\bibliographystyle{Latex/spmpsci} 
\bibliography{bib}

\clearpage

\begin{figure}
\includegraphics[scale=0.9]{figs/atp_dendro_std_fixed.pdf}
\caption{Dendrogram of distances in ATP mentality profiles for players
competing in the 2015 Grand Slams. Profiles consisted of 8 dynamic
predictors on the service game and 8 on the return game. Dissimilarity
was measured with a manhattan distance and players were clustered
using complete linkage.}
\label{fig:atp_dendro}
\end{figure}

\clearpage

\begin{figure}
\includegraphics[scale=0.9]{figs/atp_coords_std_fixed.pdf}
\caption{Parallel coordinates plot of mentality types for the ATP
  players represented in the dendrogram of
  Figure~\ref{fig:atp_dendro}. Effects were scaled to have a common
  standard deviation of one but were not centered so that the
  direction is the direction of the shift in the
  serve or return advantage.}
\label{fig:atp_coord}
\end{figure}

\clearpage

\begin{figure}
\includegraphics[scale=0.9]{figs/wta_dendro_std_fixed.pdf}
\caption{Dendrogram of distances in WTA mentality profiles for players
competing in the 2015 Grand Slams. Profiles consisted of 6 dynamic
predictors on the service game and 6 on the return game. Dissimilarity
was measured with a manhattan distance and players were clustered
using complete linkage.}
\label{fig:wta_dendro}
\end{figure}

\clearpage

\begin{figure}
\includegraphics[scale=0.9]{figs/wta_coords_std_fixed.pdf}
\caption{Parallel coordinates plot of mentality types for the WTA
  players represented in the dendrogram of
  Figure~\ref{fig:wta_dendro}. Effects were scaled to have a common
  standard deviation of one but were not centered so that the
  direction is the direction of the shift in the
  serve or return advantage.}
\label{fig:wta_coord}
\end{figure}

\clearpage

\begin{figure}
\includegraphics[scale=0.9]{figs/player_volatility.pdf}
\caption{Unexplained variance on serve and return according to the PDM
Brier scores for each player when applied to the Grand Slam validation
data. Brier scores are shown as z-scores and outliers with magnitude
of 2 or more are highlighted in blue. The ten most extreme outliers
for each tour are labeled.}
\label{fig:volatility}
\end{figure}


\end{document}
